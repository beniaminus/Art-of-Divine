\chapter{The order of the work itself.}
Now, after that we have thus orderly suited the person and his qualities, with the due circumstances of time, place, disposition of body, and substance of the matter discussed, I know not what can remain besides the main business itself, and the manner and degrees of our prosecution thereof; which, above all other, calleth for an intentive reader and resolute practice. Wherein, that we may avoid all niceness and obscurity, since we strive to profit, we will give direction for the entrance, proceeding, conclusion of this 
divine work. 

\section{The entrance into the work:}
\subsection{The common entrance, which is prayer.}
A goodly building must show some magnificence in the gate; and great personages have seemly ushers to go before them, who by their uncovered heads command reverence and way. 

Even very poets of old had wont, before their ballads, to implore the aid of their gods; and the heathen Romans entered not upon any public civil business without a solemn apprecation of good success: how much less should a Christian dare to undertake a spiritual work of such importance, not having craved the assistance of his God; which, methinks, is no less than to profess he could do well without God's leave. When we think evil, it is from ourselves; when good, from God. As prayer is our speech to God, so is each good meditation, according to Bernard God's speech to the heart; the heart must speak to God, that God may speak to it. Prayer therefore and meditation are as those famous twins in the story, or as two loving turtles, whereof separate one, the other languisheth: prayer maketh way for meditation; meditation giveth matter, strength, and life to our prayers; by which, as all other things are sanctified to us, so we are sanctified to all holy things. This is as some royal eunuch, to perfume and dress our souls, that they may be fit to converse with the King of Heaven. But the prayer that leadeth in meditation would not be long, requiring rather that the extension and length should be put into the vigour and fervency of it; for that is not here intended to be the principal business, but an introduction to another, and no otherwise than as a portal to this building of meditation. The matter whereof shall be, that the course of our meditation may be guided aright and blessed; that all distractions may be avoided, our judgment enlightened, our inventions quickened, our wills rectified, our affections whetted to heavenly things, our hearts enlarged to God-ward, our devotion, enkindled: so that we may find our corruptions abated, our graces thriven, our souls and lives every way bettered by this exercise. 

\subsection{Particular and proper entrance into the matter, which is in our choice thereof.}
Such is the common entrance into this work. There is another yet more particular and proper, wherein the mind, recollecting itself, maketh choice of that theme or matter whereupon it will bestow itself for the present, settling itself on that which it hath chosen; which is done by an inward inquisition made into our heart of what we both do and should think upon, rejecting what is unexpedient and unprofitable. In both which the soul, like unto some noble hawk, lets pass the crows and larks, and such other worthless birds that cross her way, and stoopeth upon a fowl of price, worthy of her flight ; after this manner. 

``What wilt thou muse upon, O my soul? Thou seest how little it availeth thee to wander and rove about in uncertainties; thou findest how little favour there is in these earthly things wherewith thou hast wearied thyself. Trouble not thyself any longer, with Martha, about the many and needless thoughts of the world; none but heavenly things can afford thee comfort. Up then, my soul, and mind those things that are above, whence thyself art; amongst all which, wherein shouldest thou rather meditate than of the life and glory of God's saints? A worthier employment thou canst never find, than to think upon that estate thou shalt once possess, and now desirest''

\subsection{The proceeding of our meditation; and therein a method allowed by some authors rejected by us.}
Hitherto the entrance. After which our meditation must proceed in due order, not troubledly, not preposterously. It begins in the understandings endeth in the affection; it begins in the brain, descends to the heart; begins on earth, ascends to heaven ; not suddenly, but by certain stairs and degrees, till we come to the highest. 

I have found a subtle scale of meditation, admired by some professors of this art above all other human devices, and far preferred by them to the best directions of Origen, Austin, Bernard, Hugo Bonaventure, Gerson, and whosoever hath been reputed of greatest perfection in this skill. The several stairs whereof, lest I should seem to defraud my reader through envy, I would willingly describe, were it not that I feared to scare him rather with the danger of obscurity from venturing further upon this so worthy a business; yet, lest any man perhaps might complain of an unknown loss, my margin shall find room for that which I hold too knotty for my text\footnote{See appendix \ref{appendix:scale} for ``the scale of meditation of an author, ancient but nameless.''}. In all which, after the incredible commendations of some practitioners, I doubt not but an ordinary reader will easily espy a double fault at the least, darkness and coincidence; that they are both too obscurely delivered, and that divers of them fall into other, not without some vain superfluity. For this part therefore, which concerneth the understandings I had rather to require only a deep and firm consideration of the thing propounded; which shall be done, if we follow it in our discourse through all or the principal of those places which natural reason doth afford us. Wherein, let no man plead ignorance, or fear difficulty; we are all thus far born logicians, neither is there in this so much need of skill as of industry. In which course yet we may not be too curious, in a precise search of every place and argument, without omission of any, though to be fetched in with racking the invention; for as the mind, if it go loose, and without rule, roves to no purpose; so if it be too much fettered with the gyves of strict regularity, moveth nothing at all. 

\section{Premonitions concerning our proceeding in the first part of meditation.}
Ere I enter, therefore, into any particular tractation, there are three things whereof I would premonish my reader, concerning this first part, which is in the understanding. 

First, that I desire not to bind every man to the same uniform proceeding in this part. Practice and custom may perhaps have taught other courses, more familiar and not less direct. If then we can, by any other method, work in our hearts so deep an apprehension of the matter meditated, as it may duly stir the affections, it is that only we require. 

Secondly, that whosoever applieth himself to this direction, think him not necessarily tied to the prosecution of all these logical places, which he findeth in the sequel of our treatise; so as his meditation should be lame and imperfect without the whole number: for there are some themes which will not bear all these; as, when we meditate of God, there is no room for causes or comparisons; and others yield them with such difficulty, that their search interrupteth the chief work intended. It shall be sufficient if we take the most pregnant and most voluntary.

Thirdly, that when we stick in the disposition of any of the places following, (as if, meditating of sin, I cannot readily meet with the material and formal causes, or the appendances of it,) we rack not our minds too much with the inquiry thereof; which were to strive more for logic than devotion; but, without too much disturbance of our thoughts, quietly pass over to the next. If we break our teeth with the shell, we shall find small pleasure in the kernel. 

Now then, for that my only fear is lest this part of my discourse shall seem over-perplexed unto the unlearned reader, I will, in this whole process, second my rule with his example; that so, what might seem obscure in the one may by the other be explained; and the same steps he seeth me take in this, he may accordingly tread in any other theme. 
