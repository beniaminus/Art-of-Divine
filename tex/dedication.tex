\begin{center}
    To the right worshipful\\Sir Richard Lea, Knt.\\All increase of true honour with god and men
    \end{center}
    \vspace{5pt}
    \hrule
    \vspace{6pt}
    Sir, --- Ever since I began to bestow myself upon the common good, studying wherein my labours might be most serviceable ; I still found they could be no way so well improved as in that part which concerneth devotion and the practice of true piety. For, on the one side, I perceived the number of polemical books rather to breed than end strifes; and those which are doctrinal, by reason of their multitude, rather to oppress than satisfy the reader; wherein, if we write the same things we are judged tedious; if different, singular. On the other part, respecting the reader, I saw the brains of men never more stuffed, their tongues never more stirring, their hearts never more empty, nor their hands more idle. Wherefore, after those sudden Meditations which passed me without rule\footnote{Alluding to his `Three Centuries of Meditations and Vows.' — Pratt.}, I was easily induced by their success, as a small thing moves the willing, to send forth this `Rule of Meditation;' and after my `Heaven upon Earth,' to discourse, although by way of example, of heaven above. In this Art of mine, I confess to have received more light from one obscure nameless monk, which wrote some hundred and twelve years ago, than from the directions of all other writers. I would his humility had not made him niggardly of his name, that we might have known whom to have thanked. It had been easy to have framed it with more curiosity; but God and my soul know, that I made profit the scope of my labour, and not applause; and therefore to choose, I wished rather to be rude than unprofitable. If now the simplicity of any reader shall bereave him of the benefit of my precepts, I know he may make his use of my examples. Why I have honoured it with your name, I need not give account to the world, which already knoweth your worth and deserts, and shall see by this that I acknowledge them. Go you on happily, according to the heavenly advice of your Junius, in your worthy and glorious profession; still bearing yourself as one that knoweth virtue the truest nobility, and religion the best virtue. The God whom you serve shall honour you with men, and crown you in heaven. To his grace I humbly commend you; requesting you only to accept the work, and continue your favour to the author. 
    \begin{flushright}
    Your Worship's humbly devoted\\\uppercase{Jos. Hall.}
    \end{flushright}    
    