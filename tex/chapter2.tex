\chapter{The description and kinds of meditation.}
The rather, for that whereas our divine meditation is nothing else but a bending of the mind upon some spiritual object through divers forms of discourse, until our thoughts come to an issue; and this must needs be either extemporal, and occasioned by outward occurrences offered to the mind, or deliberate and wrought out of our own heart; which again is either in matter of knowledge, for the finding out of some hidden truth, and convincing of an heresy by profound traversing of reason; or in matter of affection, for the enkindling of our love to God: the former of these two last, we, sending to the schools and masters of controversies, search after the latter; which is both of larger use, 
and such as no Christian can reject, as either unnecessary or overdifficult: for, both every Christian had need of fire put to his affections; and weaker judgments are no less capable of this divine heat, which proceeds not so much from reason as from faith. One saith, and I believe him, that God's school is more of affection than understanding: both lessons very needful, very profitable; but for this age especially the latter: for if there be some that have much zeal, little knowledge; there are more that have much knowledge without zeal: and he that hath much skill and no affection may do good to others by information of judgment, but shall never have thank, either of his own heart or of God, who useth not to cast away his love on those of whom he is but known, not loved.

\section{Concerning meditation extemporal.}
Of extemporal meditation there may be much use, no rule; forasmuch as our conceits herein vary according to the infinite multitude of objects, and their diverse manner of proffering themselves to the mind; as also for the suddenness of this act. Man is placed in this stage of the world, to view the several natures and actions of the creature; to view them, not idly, without his use, as they do him. God made all these for man, and man for his own sake. Both these purposes were lost, if man should let the creatures pass carelessly by him; only seen, not thought upon. He only can make benefit of what he sees; which if he do not, it is all one as if he were blind or brute. Whence it is that wise Solomon putteth the sluggard to school unto the ant, and our Saviour sendeth the distrustful to the lily of the field. In this kind was that meditation of the divine Psalmist;which, upon the view of the glorious frame of the heavens, was led to wonder at the merciful respect God hath to so poor a creature as man. Thus our Saviour took occasion of the water fetched up solemnly to the altar from the well of Shilo on the day of the great Hosannah, to meditate and discourse of the water of life. This holy and sweet Augustin, from occasion of the watercourse near to his lodging, running among the pebbles, sometimes more silently, sometimes in a baser murmur, and sometimes in a shriller note, entered into the thought and discourse of that excellent order which God hath settled in all these inferior things. Thus that learned and heavenly soul of our late Estye, when we sat together and heard a sweet concert of music, seemed upon this occasion carried up for the time beforehand to the place of his rest, saying, not without some passion, `` What music may we think there is in heaven!'' Thus lastly, for who knows not that examples of this kind are infinite? That faithful and reverend Deering\footnote{See Fuller's Ch. Hist. B. ix. 22.}, when the sun shined on his face, now lying on his deathbed, fell into a sweet meditation of the glory of God and his approaching joy. The thoughts of this nature are not only lawful, but so behoveful, that we cannot omit them without neglect of God, his creatures, ourselves. The creatures are half lost, if we only employ them, not learn something of them: God is wronged, if his creatures be unregarded; ourselves most of all, if we read this great volume of the creatures, and take out no lesson for our instruction. 

\section{Cautions of extemporal meditation.}
Wherein yet caution is to be had, that our meditations be not either too farfetched or savouring of superstition. Farfetched I call those which have not a fair and easy resemblance unto the matter from whence they are raised; in which case our thoughts prove loose and heartless, making no memorable impression in the mind. Superstitious, when we make choice of those grounds of meditation which are forbidden us, as teachers of vanity; or employ our own devices, though well-grounded, to an use above their reach; making them, upon our own pleasures, not only furtherances, but parts of God's worship: in both which our meditations degenerate, and grow rather perilous to the soul. Whereto add, that the mind be not too much cloyed with too frequent iteration of the same thought; which at last breeds a weariness in ourselves, and an unpleasantness of that conceit which at the first entertainment promised much delight. Our nature is too ready to abuse familiarity in any kind; and it is with meditations as with medicines, which, with over-ordinary use, lose their sovereignty, and fill instead of purging. God hath not straited us for matter, having given us the scope of the whole world; so that there is no creature, event, action, speech, which may not afford us new matter of meditation. And that which we are wont to say of fine wits, we may as truly affirm of the Christian heart, that it can make use of any thing. Wherefore, as travellers in a foreign country make every sight a lesson, so ought we in this our pilgrimage. Thou seest the heaven rolling above thy head in a constant and unmovable motion; the stars so overlooking one another, that the greatest show little, the least greatest, all glorious; the air full of the bottles of rain, or fleeces of snow, or divers forms of fiery exhalations; the sea, under one uniform face, full of strange and monstrous shapes beneath; the earth so adorned with variety of plants, that thou canst not but tread on many at once with every foot; besides the store of creatures that fly about it, walk upon it, live in it. Thou idle truant, dost thou learn nothing of so many masters? Hast thou so long read these capital letters of God's great book, and canst thou not yet spell one word of them? The brute creatures see the same things with as clear, perhaps better eyes: if thine inward eyes see not their use, as well as thy bodily eyes their shape, I know not whether is more reasonable or less brutish. 

