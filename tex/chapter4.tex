\chapter{Of the circumstances of meditation.}
From the qualities of the person we descend towards the action itself: where first we meet with those circumstances which are necessary for our predisposition to the work, place, time, site of the body. 

\section{And therein, first, of the place.}
Solitariness of place is fittest for meditation. Retire thyself from others if thou wouldest talk profitably with thyself. So Jesus meditates alone in the mount; Isaac in the fields\footnote{Genesis 24:63 ``And Isaac went out to meditate in the field at the eventide: and he lifted up his eyes, and saw, and, behold, the camels were coming.''} 
; John Baptist in the desert; David on his bed; Chrysostom\footnote{St. John Chrysostom (c.347-407), considered the most prominent doctor in the Greek Church, and the greatest preacher heard in a Christian pulpit.\cite{cath1913}} in the bath: each in several places, but all solitary. There is no place free from God, none to which he is more tied; one finds his closet most convenient, where his eyes, being limited by the known walls, call the mind, after a sort, from wandering abroad; another findeth his soul more free when it beholdeth his heaven above and about him. It matters not, so he be solitary and silent. It was a witty and divine speech of Bernard\footnote{Bernard of Cluny (or of Morlaix), a Benedictine monk of the first half of the twelfth century, poet, satirist, and hymn-writer, author of the famous verses \emph{On the Contempt of the World}\cite{cath1913}}, that the Spouse of the Soul, Christ Jesus, is bashful, neither willingly cometh to his bride in the presence of a multitude. And hence is that sweet invitation which we find of her: \emph{Come, my well beloved, let us go forth into the fields; let us lodge in the villages. Let us go up early to the vines : let us see if the vine flourish, whether it hath disclosed the first grape; or whether the pomegranates blossom: there will I give thee my love.\footnote{Song of Solomon 7:11-12 ``Come, my beloved, let us go forth into the field; let us lodge in the villages. \textsuperscript{12}Let us get up early to the vineyards; let us see if the vine flourish, whether the tender grape appear, and the pomegranates bud forth: there will I give thee my loves.''}} Abandon therefore all worldly society, that thou mayest change it for the company of God and his angels: the society, I say, of the world; not outward only, but inward also. There be many that sequester themselves from the visible company of men, which yet carry a world within them; who being alone in body, are haunted with a throng of fancies; as Jerome\footnote{St. Jerome (c.340-420), prolific early Christian writer on the Bible, theological controversies, history and letters\cite{cath1913}.}, in his wildest desert, found himself too oft in his thoughts amongst the dances of the Roman dames. This company is worse than the other; for it is more possible for some thoughtful men to have a solitary mind in the midst of a market, than for a man thus disposed to be alone in a wilderness. Both companies are enemies to meditations; whither tendeth that ancient connsel of a great master in this art, of three things requisite to this business, secresy, silence, rest: whereof the first excludeth company; the second, noise; the third, motion. It cannot be spoken how subject we are in this work to distraction; like Solomon's old man, whom the noise of every bird wakeneth\footnote{Ecclesiastes 12:4 ``And the doors shall be shut in the streets, when the sound of the grinding is low, and he shall rise up at the voice of the bird, and all the daughters of musick shall be brought low;''}. Sensual delights we are not drawn from with the threefold cords\footnote{Ecclesiastes 4:12 ``And if one prevail against him, two shall withstand him; and a threefold cord is not quickly broken.''} of judgment, but our spiritual pleasures are easily hindered. Make choice therefore of that place which shall admit the fewest occasions of withdrawing thy soul from good thoughts; wherein also even change of places is somewhat prejudicial; and I know not how it falls out, that we find God nearer us in the place where we have been accustomed familiarly to meet him: not for that his presence is confined to one place above others; but that our thoughts are, through custom, more easily gathered to the place where we have ordinarily conversed with him. 

\section{Secondly, of the time.}
One time cannot be prescribed to all: for neither is God bound to hours, neither doth the contrary disposition of men agree in one choice of opportunities. The golden hours of the morning some find fittest for meditation; when the body, newly raised, is well calmed with his late rest; and the soul hath not as yet had from these outward things any motives of alienation. Others find it best to learn wisdom of their reins in the night; hoping, with Job, that their bed will bring them comfort in their meditation; when, both all other things are still, and themselves, wearied with these earthly cares, do, out of a contempt of them, grow into greater liking and love of heavenly things. I have ever found Isaac's time fittest, who went out in the evening to meditate. No precept, no practice of others, can prescribe to us in this circumstance. It shall be enough, that, first, we set ourselves a time; secondly, that we set apart that time wherein we are aptest for this service. And as no time is prejudiced with unfitness, but every day is without difference seasonable for this work, so especially God's day. No day is barren of grace to the searcher of it; none alike fruitful to this: which being by God sanctified to himself, and to be sanctified by us to God, is privileged with blessings above others: for the plentiful instruction of that day stirreth thee up to this action, and fills thee with matter; and the zeal of thy public service warmeth thy heart to this other business of devotion. No manna fell to the Israelites on their sabbath; our spiritual manna falleth on ours most frequent. If thou wouldest haye a full soul, gather as it falls; gather it by hearing, reading, meditation: spiritual idleness is a fault this day, perhaps not less than bodily work. 

\section{Of the site and gesture of the body.}
Neither is there less variety in the site and gesture of the body; the due composedness whereof is no little advantage to this exercise. Even in our speech to God, we observe not always one and the same position: sometimes, we fall grovelling on our faces; sometimes, we bow our knees; sometimes, stand on our feet; sometimes, we lift up our hands; sometimes, cast down our eyes. God is spirit; who therefore, being a severe observer of the disposition of the soul, is not scrupulous for the body; requiring not so much that the gesture thereof should be uniform as reverent. No marvel, therefore, though in this all our teachers of meditation have commended several positions of body, according to their disposition and practice ; one, (Gerson,) sitting with the face turned up to heavenward, according to the precept of the philosopher, who taught him, that by sitting and resting the mind gathereth wisdom: another, (Gulielmus Episc. Paris\footnote{William of Auvergne (c.1190-1249), Bishop of Paris, medieval philosopher and theologian.}) leaning to some rest towards the left side, for the greater quieting of the heart: a third, (Dionys. Carthus.\footnote{Denys the Carthusian (1402-1471) born in Ryckel, a small village a few miles from Saint-Trond, whence ancient writers have often surnamed him \emph{à Ryckel}. While still a novice he had ecstasies which lasted two or three hours, and posterity has surnamed ``Doctor Ecstaticus.''\cite{cath1913}}) standing with the eyes lift up to heaven; but shut for fear of distractions. But of all other, methinketh, Isaac's choice the best, who meditated walking. In this, let every man be his own master; so be, we use that frame of body that may both testify reverence, and in some cases help to stir up further devotion; which also must needs be varied, according to the matter of our meditation. If we think of our sins, Ahab's soft pace, the publican's dejected eyes, and his hand beating his breast, are most seasonable: if of the joys of heaven, Stephen's countenance fixed above, and David's hands lift up on high, are most fitting. In all which the body, as it is the instrument and vassal of the soul, so will easily follow the affections thereof; and, in truth, then is our devotion most kindly, when the body is thus commanded his service by the spirit, and not suffered to go before it, and by his forwardness to provoke his master to emulation. 
