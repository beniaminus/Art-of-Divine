\newleadpage{meditation}{A Meditation of Death}
\meditation
\chapter{An Example Meditation, According To The Former Rules}
\section{The entrance.}
And now, my soul, that thou hast thought of the end, what can fit thee better than to think of the way? And though the forepart of the way to heaven be a good life, the latter and more immediate is death. Shall I call it the way, or the gate of life? Sure I am, that by it only we pass into that blessedness; whereof we have so thought, that we have found it cannot be thought of enough. 

\section{The description.}
What then is this death but the taking down of these sticks, whereof this earthly tent is composed? the separation of two great and old friends, till they meet again? the gaol-delivery of a long prisoner? our journey into that other worlds for which we and this thoroughfare were made? our payment of our first debt to nature; the sleep of the body and the awaking of the soul? 

\section{The division}
But, lest thou shouldest seem to flatter him whose name and face hath ever seemed terrible to others, remember that there are more deaths than one: if the first death be not so fearful as he is made, his horror lying more in the conceit of the beholder than in his own aspect, surely the second is not made so fearful as he is. No living eye can behold the terrors thereof; it is as impossible to see them, as to feel them and live. Nothing but a name is common to both. The first hath men, casualties, diseases, for his executioners; the second, devils: the power of the first is in the grave; the second, in hell: the worst of the first is senselessness; the easiest of the second is a perpetual sense of all the pain that can make a man exquisitely miserable. 

\section{The causes.}
Thou shalt have no business, O my soul, with the second death: thy first resurrection hath secured thee. Thank him that hath redeemed thee for, thy safety. And how can I thank thee enough, O my Saviour, which hast so mercifully bought off 
 