\chapter{An epilogue}
\section{Reproving the neglect; exhorting to the use of meditation}
Thus have I endeavoured, right worshipful sir, according to my slender faculty, to prescribe a method of meditation: not upon so strict terms of necessity, that whosoever goeth not my way erreth. Divers paths lead ofttimes to the same end, and every man aboundeth in his own sense. If experience and custom hath made another form familiar to any man, I forbid it not; as that learned father said of his translation, ``Let him use his own, not contemn mine.'' If any man be to choose and begin, let him practise mine till he meet with a better master. If another course may be better, I am sure this is good. Neither is it to be suffered, that, like as fantastical men, while they doubt what fashioned suit they should wear, put on nothing, so that we Christians should neglect the matter of this worthy business, while we nicely stand upon the form thereof. Wherein give me leave to complain, with just sorrow and shame, that if there be any Christian duty whose omission is notoriously shameful and prejudicial to the souls of professors, it is this of meditation. This is the very end God hath given us our souls for: we misspend them if we use them not thus. How lamentable is it, that we so employ them, as if our faculty of discourse served for nothing but our earthly provision! as if our reasonable and Christian minds were appointed for the slaves and drudges of this body, only to be the caters and cooks of our appetite! 

The world filleth us, yea cloyeth us: we find ourselves work enough to think? ``What have I yet? How may I get more? What must I lay out? What shall I leave for posterity? How 
may I prevent the wrong of mine adversary? How may I return it? What answer shall I make to such allegations? What entertainment shall I give to such friends? What courses shall I take in such suits? In what pastime shall I spend this day? In what the next? What advantage shall I reap by this practice, what loss? What was said, answered, replied, done, followed?''

Goodly thoughts, and fit for spiritual minds! Say there were no other world; how could we spend our cares otherwise? Unto this only neglect let me ascribe the commonness of that Laodicean\footnote{Lukewarm in religion, from Revelation 3:14-16 ``And unto the angel of the church of the Laodiceans write; These things saith the Amen, the faithful and true witness, the beginning of the creation of God; I know thy works, that thou art neither cold nor hot: I would thou wert cold or hot. So then because thou art lukewarm, and neither cold nor hot, I will spue thee out of my mouth.''} temper of men ; or, if that be worse, of the dead coldness which hath stricken the hearts of many, having left them nothing but the bodies of men, and vizors of Christians; to this only -- they have not meditated. It is not more impossible to live without an heart, than to be devout without meditation. Would God, therefore, my words could be in this, as the Wise Man saith the words of the wise are, like unto goads in the sides of every reader, to quicken him up, out of this dull and lazy security, to a cheerful practice of this divine meditation. Let him curse me upon his deathbed, if, looking back from thence to the bestowing of his former times, he acknowledge not these hours placed the most happily in his whole life; if he then wish not he had worn out more days in so profitable and heavenly a work! 

\begin{center}
DEO SOLI GLORIA 
\end{center}