\chapter{Of our second part of meditation; which is in the affections}
\section{Wherein is required a taste and relish of what we have thought upon.}

The most difficult and knotty part of meditation thus finished, there remaineth that, which is both more lively and more easy unto a good heart, to be wrought altogether by the affections; which if our discourses reach not unto, they prove vain and to no purpose. That which followeth therefore is the very soul of meditation, whereto all that is past serveth but as an instrument. A man is a man by his understanding part, but he is a Christian by his will and affections. 

Seeing therefore that all our former labour of the brain is only to affect the heart, after that the mind hath thus traversed the point proposed through all the heads of reason, it shall endeavour to find, in the first place, some feeling touch and sweet relish in that which it hath thus chewed; which fruit, through the blessing of God, will voluntarily follow upon a serious meditation. David saith, \emph{O taste, and see how sweet the Lord is.} In meditation we do both see and taste; but we see before we taste: sight is of the understanding; taste, of the affection: neither can we see, but we must taste; we cannot know aright, but we must needs be affected. Let the heart, therefore, first conceive and feel in itself the sweetness or bitterness of the matter meditated; which is never done without some passion, nor expressed without some hearty exclamation. 

``O blessed estate of the saints! O glory not to be expressed, even by those which are glorified! O incomprehensible salvation! What savour hath this earth to thee? Who can regard the world that believeth thee? Who can think of thee, and not be ravished with wonder and desire? Who can hope for thee, and not rejoice? Who can know thee, and not be swallowed up with admiration at the mercy of him that bestoweth thee? O blessedness, worthy of Christ's blood to purchase thee! Worthy of the continual songs of saints and angels to celebrate thee! How should I magnify thee! How should I long for thee! How should I hate all this world for thee!''
