\chapter{Of our second part of meditation; which is in the affections}
\section{Wherein is required a taste and relish of what we have thought upon.}

The most difficult and knotty part of meditation thus finished, there remaineth that, which is both more lively and more easy unto a good heart, to be wrought altogether by the affections; which if our discourses reach not unto, they prove vain and to no purpose. That which followeth therefore is the very soul of meditation, whereto all that is past serveth but as an instrument. A man is a man by his understanding part, but he is a Christian by his will and affections. 

Seeing therefore that all our former labour of the brain is only to affect the heart, after that the mind hath thus traversed the point proposed through all the heads of reason, it shall endeavour to find, in the first place, some feeling touch and sweet relish in that which it hath thus chewed; which fruit, through the blessing of God, will voluntarily follow upon a serious meditation. David saith, \emph{O taste, and see how sweet the Lord is\footnote{Psalms 34:8 ``O taste and see that the LORD is good: blessed is the man that trusteth in him.''}.} In meditation we do both see and taste; but we see before we taste: sight is of the understanding; taste, of the affection: neither can we see, but we must taste; we cannot know aright, but we must needs be affected. Let the heart, therefore, first conceive and feel in itself the sweetness or bitterness of the matter meditated; which is never done without some passion, nor expressed without some hearty exclamation. 

``O blessed estate of the saints! O glory not to be expressed, even by those which are glorified! O incomprehensible salvation! What savour hath this earth to thee? Who can regard the world that believeth thee? Who can think of thee, and not be ravished with wonder and desire? Who can hope for thee, and not rejoice? Who can know thee, and not be swallowed up with admiration at the mercy of him that bestoweth thee? O blessedness, worthy of Christ's blood to purchase thee! Worthy of the continual songs of saints and angels to celebrate thee! How should I magnify thee! How should I long for thee! How should I hate all this world for thee!''

\section{Secondly, a complaint, bewailing cur wants and untowardness.}

After this taste shall follow a complaint, wherein the heart bewaileth to itself his own poverty, dulness, and imperfection; chiding and abasing itself in respect of his wants and indisposition: wherein humiliation truly goeth before glory; for the more we are cast down in our conceit, the higher shall God lift us up at the end of this exercise in spiritual rejoicing. 

``But alas! Where is my love? Where is my longing? Where art thou, O my soul? What heaviness hath overtaken thee? How hath the world bewitched and possessed thee, that thou art become so careless of thy home, so senseless of spiritual delights, so fond upon these vanities? Dost thou doubt whether there be an heaven? Or whether thou have a God and a Saviour there? O far be from thee this atheism: far be from thee the least thought of this desperate impiety. Woe were thee, if thou believedst not! But, O thou of little faith, dost thou believe there is happiness, and happiness for thee; and desirest it not, and delightest not in it? Alas! How weak and unbelieving is thy belief! How cold and faint are thy desires! Tell me, what such goodly entertainment hast thou met withal here on earth that was worthy to withdraw thee from these heavenly joys? What pleasure in it ever gave thee contentment? Or what cause of dislike findest thou above? O no, my soul, it is only thy miserable drowsiness, only thy security; the world, the world hath besotted thee, hath undone thee with carelessness. Alas! If thy delight be so cold, what difference is there in thee from an ignorant heathen, that doubts of another life? Yea, from an epicure, that denies it? Art thou a Christian, or art thou none? If thou be what thou professest, away with this dull and senseless worldliness; away with this earthly uncheerfulness; shake off at last this profane and godless security, that hath thus long weighed thee down from mounting up to thy joys. Look up to thy God and to thy crown, and say with confidence, \emph{O Lord, I have waited for thy salvation\footnote{Genesis 49:18 ``I have waited for thy salvation, O LORD.''}.}''

\section{An hearty wish of the soul for what it complaineth to want.}

After this complaint must succeed an hearty and passionate wish of the soul, which ariseth clearly from the two former degrees; for that which a man hath found sweet and comfortable, and complains that he still wanteth, he cannot but wish to enjoy. 

``O Lord, that I could wait and long for thy salvation! O that I could mind the things above! That, as I am a stranger indeed, so I could be also in affection! O that mine eyes, like the eyes of the first martyr, could, by the light of faith, see but a glimpse of heaven! O that my heart could be rapt up thither in desire! How should I trample upon these poor vanities of the earth! How willingly should I endure all sorrows, all torments! How scornfully should I pass by all pleasures! How should I be in travail of my dissolution! O when shall that blessed day come, when, all this wretched worldliness removed, I shall solace myself in my God? \emph{Behold, as the hart brayeth for the rivers of waters, so panteth my soul after thee, O God: my soul thirsteth for God, even for the living God: O, when shall I come and appear before the presence of God?\footnote{Lewis Bayly, The Practice of Piety: Directing a Christian How to Walk, that He May Please God, from the section ``Things to be Meditated on as Thou Goest to the Church''  }}'' 

\section{An humble confession of our disability to effect what we wish.}

After this wishing shall follow humble confession, by just order of nature; for having bemoaned our want, and wished supply, not finding this hope in ourselves, we must needs acknowledge it to him, of whom only we may both seek and find; where it is to be duly observed, how the mind is by turns depressed and lifted up; being lifted up with our taste of joy, it is cast down with complaint; lift up with wishes, it is cast down with confession: which order doth best hold it in sure and just temper, and maketh it more feeling of the comfort which followeth in the conclusion. This confession must derogate all from ourselves, and ascribe all to God. 

``Thus I desire, Lord, to be aright affected towards thee and thy glory. I desire to come to thee; but, alas! How weakly, how heartlessly! Thou knowest that I can neither come to thee, nor desire to come, but from thee. It is nature that holds me from thee: this treacherous nature favours itself; loveth the world; hateth to think of a dissolution; and chooseth rather to dwell in this dungeon with continual sorrow and complaint, than to endure a parting, although to liberty and joy. Alas, Lord, it is my misery that I love my pain! How long shall these vanities thus besot me? It is thou only that canst turn away mine eyes from regarding these follies, and my heart from affecting them: thou only, who, as thou shalt one day receive my soul into heaven, so now beforehand canst fix my soul upon heaven and thee." 

\section{An earnest petition for that which we confess to want.}

After confession, naturally follows petition; earnestly requesting that at his hands, which we acknowledge ourselves unable, and none but God able to perform. 

``O carry it up, therefore, thou that hast created and redeemed it, carry it up to thy glory. O let me not always be thus dull and brutish: let not these scales of earthly affection always dim and blind mine eyes. O thou that layedst clay upon the blind man's eyes, take away this clay from mine eyes; wherewith, alas! They are so daubed up, that they cannot see heaven. Illuminate them from above, and in thy light let me see light. O thou that hast prepared a place for my soul, prepare my soul for that place; prepare it with holiness; prepare it with desire; and even while it sojourneth on earth let it dwell in heaven with thee, beholding ever the beauty of thy face, the glory of thy saints, and of itself.''

\section{A vehement enforcement of our petition.}

After petition, shall follow the enforcement of our request, from argument and importunate obsecration: wherein we must take heed of complimenting in terms with God; as knowing that he will not be mocked by any fashionable form of suit, but requireth holy and feeling entreaty. 

``How graciously hast thou proclaimed to the world, that whoever wants wisdom shall ask it of thee, which neither deniest nor upbraidest! O Lord, I want heavenly wisdom, to conceive aright of heaven: I want it, and ask it of thee: give me to ask it instantly; and give me, according to thy promise, abundantly. Thou seest it is no strange favour that I beg of thee: no other than that which thou hast richly bestowed upon all thy valiant martyrs, confessors, servants, from the beginning; who never could have so cheerfully embraced death and torment, if, through the midst of their flames and pain, they had not seen their crown of glory. The poor thief on the cross had no sooner craved thy remembrance when thou camest to thy kingdom, than thou promisedst to take him with thee into heaven. Presence was better to him than remembrance. Behold, now thou art in thy kingdom; I am on earth: remember thine unworthy servant; and let my soul, in conceit, in affection, in conversation, be this day and for ever with thee in paradise. I see, \emph{man walketh in a vain shadow, and disquieteth himself in vain\footnote{Psalms 39:6 ``Surely every man walketh in a vain shew: surely they are disquieted in vain: he heapeth up riches, and knoweth not who shall gather them.''}:} they are pitiful pleasures he enjoyeth, while he forgetteth thee: I am as vain; make me more wise: O let me see heaven; and I know I shall never envy nor follow them. \emph{My times are in thy hand\footnote{Psalms 31:15 ``My times are in thy hand: deliver me from the hand of mine enemies, and from them that persecute me.''}:} I am no better than my fathers; a stranger on earth. As I speak of them, so the next, yea this generation shall speak of me, as one that was. My life is a bubble, a smoke, a shadow, a thought: I know it is no abiding in this thoroughfare: O suffer me not so mad, as, while I pass on the way, I should forget the end. It is that other life that I must trust to: with thee it is that I shall continue: O let me not be so foolish as to settle myself on what I must leave, and to neglect eternity. I have seen enough of this earth; and yet I love it too much: O let me see heaven another while; and love it so much more than the earth, by how much the things there are more worthy to be loved. O God, look down on thy wretched pilgrim, and teach me to look up to thee, and to see thy goodness in the land of the living. Thou, that boughtest heaven for me, guide me thither; and, for the price that it cost thee, for thy mercies, sake, in spite of all temptations, enlighten thou my soul, direct it, crown it.''

\section{A cheerful confidence of obtaining what we have requested and enforced.}

After this enforcement doth follow confidence; wherein the soul, after many doubtful and unquiet bickerings, gathereth up her forces, and cheerfully rouseth up itself; and, like one of David's worthies, breaketh through a whole army of doubts, and fetcheth comfort from the well of life; which, though in some later, yet in all, is a sure reward from God of sincere meditation. 

``Yea, be then bold, O my soul; and do not merely crave, but challenge this favour of God, as that which he oweth thee; he oweth it thee, because he hath promised it; and by his mercy hath made his gift his debt: \emph{Faithful is he that hath promised, which will also do it\footnote{1 Thessalonians 5:24 ``Faithful is he that calleth you, who also will do it.''}.} Hath he not given thee not only his hand in the sweet hopes of the gospel, but his seal also in the sacraments? Yea, besides promise, hand, seal, hath he not given thee a sure earnest of thy salvation in some weak but true graces? Yet more, hath he not given thee, besides earnest, possession; while he, that is the \emph{truth} and \emph{life}, saith \emph{He that believeth hath everlasting life, and hath passed from death to life?\footnote{John 5:24 ``Verily, verily, I say unto you, He that heareth my word, and believeth on him that sent me, hath everlasting life, and shall not come into condemnation; but is passed from death unto life.''}} Canst thou not then be content to cast thyself upon this blessed issue; if God be merciful, I am glorious: I have thee already, O my life? God is faithful, and I do believe: who shall separate me from the love of Christ? From my glory with Christ? Who shall pull me out of my heayen? Go to then, and \emph{return to thy rest, O my soul\footnote{Psalms 116:7 ``Return unto thy rest, O my soul; for the LORD hath dealt bountifully with thee.''}:} make use of that heaven wherein thou art, and be happy.''

Thus we have found that our meditation, like the wind, gathereth strength in proceeding; and as natural bodies the nearer they come to their places move with more celerity, so doth the soul in this course of meditation, to the unspeakable benefit of itself. 