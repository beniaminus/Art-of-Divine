\chapter{Joseph Hall}
\label{appendix:bio}

Joseph Hall (1574-1656)\cite{brit1911}, English bishop and satirist, was born at Bristow park, near Ashby de la Zouch, Leicestershire, on the 1st of July 1574. His father, John Hall, was agent in the town for Henry, earl of Huntingdon, and his mother, Winifred Bambridge, was a pious lady, whom her son compared to St Monica. Joseph Hall received his early education at the local school, and was sent (1589) to Emmanuel College, Cambridge. Hall was chosen for two years in succession to read the public lecture on rhetoric in the schools, and in 1595 became fellow of his college. During his residence at Cambridge he wrote his \emph{Virgidemiarum} (1597), satires written after Latin models. The claim he put forward in the prologue to be the earliest English satirist:---
\begin{verse}
    ``I first adventure, follow me who list\\
    And be the second English satirist''---
\end{verse}
gave bitter offence to John Marston, who attacks him in the satires published in 1598. The archbishop of Canterbury gave an order (1599) that Hall's satires should be burnt with works of John Marston, Marlowe, Sir John Davies and others on the ground of licentiousness, but shortly afterwards Hall's book, certainly unjustly condemned, was ordered to be ``staied at the press,'' which may be interpreted as reprieved (see Notes and Queries, 3rd series, xii. 436). Having taken holy orders, Hall was offered the mastership of Blundell's school, Tiverton, but he refused it in favour of the living of Halsted, Essex, to which he was presented (1601) by Sir Robert Drury. In his parish he had an opponent in a Mr Lilly\footnote{Probably John Lilly, a dramatist, author of ``Euphues, the Anatomy of Wit,'' etc.} , whom he describes as ``a witty and bold atheist.'' In 1603 he married; and in 1605 he accompanied Sir Edmund Bacon to Spa, with the special aim, he says, of acquainting himself with the state and practice of the Romish Church. At Brussels he disputed at the Jesuit College on the authentic character of modern miracles, and his inquiring and argumentative disposition more than once threatened to produce serious results, so that his patron at length requested him to abstain from further discussion. His devotional writings had attracted the notice of Henry, prince of Wales, who made him one of his chaplains (1608). In 1612 Lord Denny, afterwards Earl of Norwich, gave him the curacy of Waltham-Holy-Cross, Essex, and in the same year he received the degree of D.D. Later he received the prebend of Willenhall in the collegiate church of Wolverhampton, and in 1616 he accompanied James Hay, Lord Doncaster, afterwards earl of Carlisle, to France, where he was sent to congratulate Louis XIII. on his marriage, but Hall was compelled by illness to return. In his absence the king nominated him dean of Worcester, and in 1617 he accompanied James to Scotland, where he defended the five points of ceremonial which the king desired to impose upon the Scots. In the next year he was one of the English 848 deputies at the synod of Dort. In 1624 he refused the see of Gloucester, but in 1627 became bishop of Exeter.

He took an active part in the Arminian and Calvinist controversy in the English church. He did his best in his \emph{Via media, The Way of Peace,} to persuade the two parties to accept a compromise. In spite of his Calvinistic opinions he maintained that to acknowledge the errors which had arisen in the Catholic Church did not necessarily imply disbelief in her catholicity, and that the Church of England having repudiated these errors should not deny the claims of the Roman Catholic Church on that account. This view commended itself to Charles I. and his episcopal advisers, but at the same time Archbishop Laud sent spies into Hall's diocese to report on the Calvinistic tendencies of the bishop and his lenience to the Puritan and low-church clergy. Hall says he was thrice down on his knees to the King to answer Laud's accusations and at length threatened to ``cast up his rochet'' rather than submit to them. He was, however, amenable to criticism, and his defence of the English Church, entitled \emph{Episcopacy by Divine Right} (1640), was twice revised at Laud's dictation. This was followed by \emph{An Humble Remonstrance to the High Court of Parliament} (1640 and 1641), an eloquent and forceful defence of his order, which produced a retort from the syndicate of Puritan divines, who wrote under the name of ``Smectymnuus,'' and was followed by a long controversy to which Milton contributed five pamphlets, virulently attacking Hall and his early satires.

In 1641 Hall was translated to the see of Norwich, and in the same year sat on the Lords' Committee on religion. On the 30th of December he was, with other bishops, brought before the bar of the House of Lords to answer a charge of high treason of which the Commons had voted them guilty. They were finally convicted of an offence against the Statute of Praemunire, and condemned to forfeit their estates, receiving a small maintenance from the parliament. They were immured in the Tower from New Year to Whitsuntide, when they were released on finding bail for £5000 each. On his release Hall proceeded to his new diocese at Norwich, the revenues of which he seems for a time to have received, but in 1643, when the property of the ``malignants'' was sequestrated, Hall was mentioned by name. Mrs Hall had difficulty in securing a fifth of the maintenance (£400) assigned to the bishop by the parliament; they were eventually ejected from the palace, and the cathedral was dismantled. Hall retired to the village of Higham, near Norwich, where he spent the time preaching and writing until ``he was first forbidden by man, and at last disabled by God.'' He bore his many troubles and the additional burden of much bodily suffering with sweetness and patience, dying on the 8th of September 1656. Thomas Fuller says: ``He was commonly called our English Seneca, for the purenesse, plainnesse, and fulnesse of his style. Not unhappy at Controversies, more happy at Comments, very good in his Characters, better in his Sermons, best of all in his Meditations.''

Bishop Hall's polemical writings, although vigorous and effective, were chiefly of ephemeral interest, but many of his devotional writings have been often reprinted. It is by his early work as the censor of morals and the unsparing critic of contemporary literary extravagance and affectations that he is best known. \emph{Virgidemiarum. Sixe Bookes. First three Bookes. Of Toothlesse Satyrs. (1) Poeticall, (2) Academicall, (3) Morall} (1597) was followed by an amended edition in 1598, and in the same year by \emph{Virgidemiarum. The three last bookes. Of byting Satyres} (reprinted 1599). His claim to be reckoned the earliest English satirist, even in the formal sense, cannot be justified. Thomas Lodge, in his \emph{Fig for Momus} (1593), had written four satires in the manner of Horace, and John Marston and John Donne both wrote satires about the same time, although the publication was in both cases later than that of \emph{Virgidemiae.} But if he was not the earliest, Hall was certainly one of the best. He writes in the heroic couplet, which he manoeuvres with great ease and smoothness. In the first book of his satires (\emph{Poeticall}) he attacks the writers whose verses were devoted to licentious subjects, the bombast of \emph{Tamburlaine} and tragedies built on similar lines, the laments of the ghosts of the \emph{Mirror for Magistrates,} the metrical eccentricities of Gabriel Harvey and Richard Stanyhurst, the extravagances of the sonneteers, and the sacred poets (Southwell is aimed at in ``Now good St Peter weeps pure Helicon, And both the Mary's make a music moan''). In Book II. Satire 6 occurs the well-known description of the trencher-chaplain, who is tutor and hanger-on in a country manor. Among his other satirical portraits is that of the famished gallant, the guest of ``Duke Humfray.''\footnote{The tomb of Sir John Beauchamp (d. 1358) in old St Paul's was commonly known, in error, as that of Duke Humphrey of Gloucester. ``To dine with Duke Humphrey'' was to go hungry among the debtors and beggars who frequented ``Duke Humphrey's Walk'' in the cathedral.} Book VI. consists of one long satire on the various vices and follies dealt with in the earlier books. If his prose is sometimes antithetical and obscure, his verse is remarkably free from the quips and conceits which mar so much contemporary poetry.

He also wrote \emph{The King’s Prophecie; or Weeping Joy} (1603), a gratulatory poem on the accession of James I.; \emph{Epistles,} both the first and second volumes of which appeared in 1608 and a third in 1611; \emph{Characters of Virtues and Vices} (1608), versified by Nahum Tate (1691); \emph{Solomons Divine Arts} \ldots\ (1609); and, probably \emph{Mundus alter et idem sive Terra Australis antehac semper incognita \ldots\ lustrata} (1605? and 1607), by ``Mercurius Britannicus,'' translated into English by John Healy (1608) as \emph{The Discovery of a New World or A Description of the South Indies \ldots\ by an English Mercury}. \emph{Mundus alter} is an excuse for a satirical description of London, with some criticism of the Romish church, its manners and customs, and is said to have furnished Swift with hints for \emph{Gulliver's Travels}. It was not ascribed to him by name until 1674, when Thomas Hyde, the librarian of the Bodleian, identified ``Mercurius Britannicus'' with Joseph Hall. For the question of the authorship of this pamphlet, and the arguments that may be advanced in favour of the suggestion that it was written by Alberico Gentili, see E. A. Petherick, \emph{Mundus alter et idem,} reprinted from the \emph{Gentleman’s Magazine} (July 1896). His controversial writings, not already mentioned, include:--- \emph{A Common Apology \ldots\ against the Brownists} (1610), in answer to John Robinson’s \emph{Censorious Epistle;} \emph{The Olde Religion: A treatise, wherein is laid downe the true state of the difference betwixt the Reformed and the Romane Church; and the blame of this schisme is cast upon the true Authors} \ldots\ (1628); \emph{Columba Noae olivam adferens} \ldots, a sermon preached at St Paul's in 1623; \emph{Episcopacie by Divine Right} (1640); \emph{A Short Answer to the Vindication of Smectymnuus} (1641); \emph{A Modest Confutation of \ldots\ (Milton's) Animadversions} (1642).

His devotional works include:--- \emph{Holy Observations Lib. I. Some few of David’s Psalmes Metaphrased} (1607 and 1609); three centuries of \emph{Meditations and Vowes, Divine and Morall} (1606, 1607, 1609), edited by Charles Sayle (1901); \emph{The Arte of Divine Meditation} (1607); \emph{Heaven upon Earth, or of True Peace and Tranquillitie of Mind} (1606), reprinted with some of his letters in John Wesley's \emph{Christian Library,} vol. iv. (1819); \emph{Occasional Meditations} \ldots\ (1630), edited by his son Robert Hall; \emph{Henochisme; or a Treatise showing how to walk with God} (1639), translated from Bishop Hall's Latin by Moses Wall; \emph{The Devout Soul; or Rules of Heavenly Devotion} (1644), often since reprinted; \emph{The Balm of Gilead} \ldots\ (1646, 1752); \emph{Christ Mysticall; or the blessed union of Christ and his Members} (1647), of which General Gordon was a student (reprinted from Gordon's copy, 1893); \emph{Susurrium cum Deo} (1659); \emph{The Great Mysterie of Godliness} (1650); \emph{Resolutions and Decisions of Divers Practicall cases of Conscience} (1649, 1650, 1654).

The chief authority for Hall's biography is to be found in his autobiographical tracts: \emph{Observations of some Specialities of Divine Providence in the Life of Joseph Hall, Bishop of Norwich, Written with his own hand;} and his \emph{Hard Measure,} a reprint of which may be consulted in Dr Christopher Wordsworth's \emph{Ecclesiastical Biography}. The best criticism of his satires is to be found in Thomas Warton's \emph{History of English Poetry,} vol. iv. pp. 363-409 (ed. Hazlitt, 1871), where a comparison is instituted between Marston and Hall. In 1615 Hall published \emph{A Recollection of such treatises as have been \ldots\ published \ldots\ }(1615, 1617, 1621); in 1625 appeared his \emph{Works} (reprinted 1627, 1628, 1634, 1662). The first complete \emph{Works} appeared in 1808, edited by the Rev. Josiah Pratt. Other editions are by Peter Hall (1837) and by Philip Wynter (1863). See also \emph{Bishop Hall, his Life and Times} (1826), by Rev. John Jones; \emph{Life of Joseph Hall,} by Rev. George Lewis (1886); A. B. Grosart, \emph{The Complete Poems of Joseph Hall \ldots\ with introductions, etc.} (1879); \emph{Satires, etc.} (\emph{Early English Poets,} ed. S. W. Singer, 1824). Many of Hall's works were translated into French, and some into Dutch, and there have been numerous selections from his devotional works.