\chapter{Of the matter and subject of our meditation.}
Now time and order call us from these circumstances to the matter and subject of meditation: which must be divine and spiritual, not evil nor worldly. O the carnal and unprofitable thoughts of men! We all meditate: one, how to do ill to others; another, how to do some earthly good to himself; another to hurt himself under a colour of good; as how to accomplish his lewd desires, the fulfilling whereof proveth the bane of the soul; how he may sin unseen, and go to hell with the least noise of the world. Or perhaps some better minds bend their thoughts upon the search of natural things; the motions of every heaven and of every star; the reason and course of the ebbing and flowing of the sea; the manifold kinds of simples that grow out of the earth, and creatures that creep upon it, with all their strange qualities and operations; or perhaps the several forms of government and rules of state take up their busy heads: so that, while they would be acquainted with the whole world, they are strangers at home; and while they seek to know all other things, they remain unknown of themselves. The God that made them, the vileness of their nature, the danger of their sins, the multitude of their imperfections, the Saviour that bought them, the heaven that he bought for them, are in the mean time as unknown, as unregarded, as if they were not. Thus do foolish children spend their time and labour in turning over leaves to look for painted babes, not at all respecting the solid matter under their hands. We fools, when will we be wise, and, turning our eyes from vanity, with that sweet singer of Israel, make God's statutes our song and meditation in the house of our pilgrimage? Earthly things proffer themselves with importunity; heavenly things must with importunity be sued to. 

Those, if they were not so little worth would not be so forward, and being forward need not any meditation to solicit them; these, by how much more hard they are to entreat, by so much more precious they are being obtained, and therefore worthier our endeavour. As then we cannot go amiss so long as we keep ourselves in the track of divinity, while the soul is taken up with the thoughts either of the Deity in his essence and persons, (sparingly yet in this point, and more in faith and admiration than inquiry,) or of his attributes, his justice, power, wisdom, mercy, truth; or of his works, in the creation, preservation, government of all things; according to the Psalmist, \emph{I will meditate of the beauty of thy glorious Majesty, and thy wonderful works\footnote{Psalms 145:5 ``I will speak of the glorious honour of thy majesty, and of thy wondrous works.''};} so most directly in our way, and best fitting our exercise of meditation, are those matters in divinity which can most of all work compunction in the heart, and most stir us up to devotion. Of which kind are the meditations concerning Christ Jesus our Mediator; his incarnation, miracles, life, passion, burial, resurrection, ascension, intercession; the benefit of our redemption, the certainty of our election, the graces and proceeding of our sanctification, our glorious estate in paradise lost in our first parents, our present vileness, our inclination to sin, our several actual offences, the temptations and sleights of evil angels, the use of the sacraments, nature and practice of faith and repentance, the miseries of our life, with the frailty of it, the certainty and uncertainty of our death, the glory of God's saints above, the awfulness of judgment, the terrors of hell; and the rest of this quality; wherein both it is fit to have variety, for that even the strongest stomach doth not always delight in one dish, and yet so to change that our choice may be free from wildness and inconstancy. 

