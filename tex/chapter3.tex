\chapter{Of meditation deliberate.}

\section{Wherein, first, the qualities of the person:}
\subsection{Of whom is required, first, that he be pure from his sins.}

Deliberate meditation is that we chiefly inquire for; which both may be well guided, and shall be not a little furthered by precepts: part whereof the labours of others shall yield us; and part, the plainest mistress, experience. 

Wherein order requires of us, first, the qualities of the person fit for meditation; then the circumstances, manner, and proceedings of the work. 

The hill of meditation may not be climbed with a profane foot: but, as in the delivery of the Law, so here, no beast may touch God's hill, lest he die; only the pure of heart have promise to see God. Sin dimmeth and dazzleth the eye, that it cannot behold spiritual things. The guard of heavenly soldiers was about Elisha's servant, before: he saw them not before, through the scales of his infidelity. The soul must therefore be purged ere it can profitably meditate. And as of old they were wont to search for and thrust out malefactors from the presence, ere they went to sacrifice; so must we our sins, ere we offer our thoughts to God. First, saith David, \emph{I will wash my hands in innocence then I will compass thine altar\footnote{Psalms 26:6 ``I will wash mine hands in innocency: so will I compass thine altar, O LORD:''}.} Whereupon, not unfitly, did that worthy chancellor of Paris make the first stair of his ladder of contemplation humble repentance. The cloth that is white, which is wont to be the colour of innocence, is capable of any dye; the black, of none other. Not that we require an absolute perfection; which, as it is incident unto none, so if it were, would exclude all need and use of meditation; but rather an honest sincerity of the heart, not willingly sinning, willingly repenting when we have sinned: which whoso finds in himself, let him not think any weakness a lawful bar to meditation. He that pleads this excuse is like some simple man, which, being half starved with cold, refuseth to come near the fire, because he findeth not heat enough in himself. 

\subsection{Secondly, that he be free from worldly thoughts.}

Neither may the soul that hopeth to profit by meditation suffer itself for the time entangled with the world, which is all one as to come to God's flaming bush on the hill of visions with our shoes on our feet. Thou seest the bird whose feathers are limed unable to take her former flight; so are we, when our thoughts are clinged together by the worlds to soar up to our heaven in meditation. The pair of brothers must leave their nets if they will follow Christ; Elisha his oxen, if he will attend a prophet. It must be a free and a light mind that can ascend this mount of contemplation, overcoming this height, this steepness. Cares are an heavy load and uneasy; these must be laid down at the bottom of this hill if we ever look to attain the top. Thou art loaded with household cares, perhaps public; I bid thee not cast them away; even these have their season, which thou canst not omit without impiety; I bid thee lay them down at thy closet door when thou attemptest this work. Let them in with thee, thou shalt find them troublesome companions, ever distracting thee from thy best errand. Thou wouldest think of heaven, thy barn comes in thy way; or perhaps thy 'count book, or thy coffers; or, it may be, thy mind is beforehand travelling up on the morrow's journey. So while thou thinkest of many things, thou thinkest of nothing; while thou wouldest go many ways, thou standest still. 

And as in a crowd, while many press forward at once through one door none proceedeth; so when variety of thoughts tumultuously throng in upon the mind, each proveth a bar to the other, and all an hinderance to him that entertains them.

\subsection{Thirdly that he be constant;}
\subsubsection{And that, first, in time and matter.}
And as our client of meditation must both be pure and free in undertaking this task, so also constant in continuing it; constant both in time and in matter; both in a set course and hour reserved for this work, and in an unwearied prosecution of it once begun. Those that meditate by snatches and uncertain fits, when only all other employments forsake them, or when good motions are thrust upon them by necessity, let them never hope to reach to any perfection; for these feeble beginnings of lukewarm grace, which are wrought in them by one fit of serious meditation, are soon extinguished by intermission, and by mis-wonting perish. This day's meal, though large and liberal, strengthens thee not for tomorrow; the body languisheth if there be not a daily supply of repast. Thus feed thy soul by meditation. Set thine hours and keep them, and yield not to an easy distraction. There is no hardness in this practice but in the beginning; use shall give it, not ease only, but delight. Thy companion entertaineth thee this while in loving discourses, or some unexpected business offers to interrupt thee; never any good work shall want some hinderance; either break through the lets, except it be with incivility or loss; or if they be importunate, pay thyself the time that was unseasonably borrowed, and recompense thine omitted hours with the double labours of another day. For thou shalt find that deferring breeds, besides the loss, an indisposition to good; so that what was before pleasant to thee, being omitted, tomorrow grows harsh, the next day unnecessary, afterward odious. Today thou canst, but wilt not; tomorrow thou couldest, but listest not; the next day thou neither wilt nor canst bend thy mind on these thoughts. So I have seen friends, that upon neglect of duty grow overly; upon overliness, strange; upon strangeness, to utter defiance. Those whose very trade is divinity, methinks, should omit no day without his line of meditation; those which are secular men, not many; remembering that they have a common calling of Christianity to attend, as well as a special vocation in the world; and that other, being more noble and important may justly challenge both often and diligent service. 

\subsubsection{Secondly, that he be constant in the continuance.}
And as this constancy requires thee to keep day with thyself, unless thou wilt prove bankrupt in good exercises; so also that thy mind should dwell upon the same thought without flitting, without, weariness, until it have attained to some issue of spiritual profit; otherwise it attempteth much, effecteth nothing. What availeth it to knock at the door of the heart, if we depart ere we have an answer? What are we the warmer if we pass hastily along by the hearth and stay not at it? Those that do only travel through Africa become not blackamoors; but those which are born there, those that inhabit there. We account those damsels too light of their love which betrothe themselves upon the first sight, upon the first motion; and those we deem of much price which require long and earnest soliciting. He deceiveth himself that thinketh grace so easily won; there must be much suit and importunity ere it will yield to our desires. Not that we call for a perpetuity of this labour of meditation; human frailty could never bear so great a toil. Nothing under heaven is capable of a continual motion without complaint; it is enough for the glorified spirits above to be ever thinking and never weary. The mind of man is of a strange metal; if it be not used, it rusteth; if used hardly, it breaketh: briefly, it is sooner dulled than satisfied with a continual meditation. 

Whence it came to pass that those ancient monks who intermeddled bodily labour with their contemplations proved so excellent in this divine business; when those at this day which having mewed and mured up themselves from the world, spend themselves wholly upon their beads and crucifix pretending no other work but meditation have cold hearts to God, and to the world show nothing but a dull shadow of devotion; for that, if the thoughts of these latter were as divine as they are superstitious, yet being without all interchangeableness bent upon the same discourse, the mind must needs grow weary, the thoughts remiss and languishing, the objects tedious; while the other refreshed themselves with this wise variety; employing the hands while they called off the mind, as good comedians do mix their parts, that the pleasantness of the one may temper the austereness of the other; whereupon they gained both enough to the body, and to the soul more than if it had been all the while busied. Besides, the excellency of the object letteth this assiduity of meditation, which is so glorious, that, like unto the sun, it may abide to have an eye cast upon it for a while, will not be gazed upon; whosoever ventureth so far, loseth both his hope and his wits. 

If we hold with that blessed Monica\footnote{St. Monica (333-387), the mother of St. Augustine.\cite{cath1913}}, that such like cogitations are the food of the mind; yet even the mind also hath her satiety, and may surfeit of too much. It shall be sufficient therefore that we persevere in our meditation without any such affectation of perpetuity, and leave without a light fickleness; making always not our hour-glass, but some competent increase of our devotion, the measure of our continuance; knowing that, as for heaven, so for our pursuit of grace, it shall avail us little to have begun well without perseverance; and withal, that the soul of man is not always in the like disposition, but sometimes is longer in settling, through some unquietness or more obstinate distraction; sometimes heavier, and sometimes more active and nimble to despatch. Gerson\footnote{Jean de Charlier de Gerson (1363-1429), chancellor of the University of Paris, representative of the French king to the Council of Constance, and known as ``the soul of the council''\cite{cath1913}}, whose authority (saving our just quarrel against him for the Council of Constance\footnote{Alluding probably to the active part he took at the Council against Jerome of Prague.\cite{lenf1714}}), I rather use because our adversaries disclaim him for theirs, professeth he hath been sometimes four hours together working his heart ere he could frame it to purpose; a singular pattern of unwearied constancy, of an unconquerable spirit, whom his present unfitness did not so much discourage as it whetted him to strive with himself till he could overcome. And surely other victories are hazardous; this certain if we will persist to strive: other fights are upon hope; this upon assurance, while our success dependeth upon the promise of God, which cannot disappoint us. Persist therefore, and prevail; persist till thou hast prevailed; so that which thou begannest with difficulty shall end in comfort. 
